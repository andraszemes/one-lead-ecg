\documentclass[titlepage,12pt]{article}
\usepackage[left=3cm, right=2.5cm, top=2.5cm]{geometry}
\usepackage{mathptmx}
\usepackage[utf8]{inputenc}
\usepackage{fancyhdr}
\usepackage{titling}
\usepackage{tabto}
\usepackage[ddmmyyyy]{datetime}
\usepackage{graphicx}
\usepackage{listings}
\usepackage{siunitx}
\usepackage[bottom]{footmisc}
\usepackage{enumitem}
\renewcommand{\dateseparator}{.}
\renewcommand{\figurename}{Obrázok}
\renewcommand{\tablename}{Tabuľka}
\renewcommand{\contentsname}{Obsah}
\graphicspath{ {/home/andras/Documents/kop/docs/img/} }
\title{\textbf{Jednozvodový elektrokardiograf} \\
\large Praktická časť odbornej zložky maturitnej skúšky}
\date{\empty}
\setcounter{section}{-1}

\begin{document}
\newgeometry{bottom=5cm} 
\bgroup
	\fancypagestyle{empty} {
		\fancyhead[C] {Stredná priemyselná škola elektrotechnická S. A. Jedlika v Nových Zámkoch}
		\fancyfoot[L] {
			\begin{flushleft}
				Nové Zámky, 2019
			\end{flushleft}		
		}
		\fancyfoot[R] {
			\begin{flushright}
				riešiteľ: \textbf{András Zemes} \\
				ročník štúdia: \textbf{štvrtý} \\
				\vtop{\vskip-.1cm \hrulefill}
				konzultant: Mgr. Peter Hudec
			\end{flushright}
		}
		\fancyfoot[C] {\empty}		
	}
	\maketitle
\egroup

\restoregeometry

\setcounter{page}{2}
\section*{Praktická časť odbornej zložky maturitnej skúšky}
\underline{Zadanie úlohy pre komplexnú maturitnú skúšku:} 
\newline

\begin{description}
	\item [Meno a priezvisko:]
		\tabto{5cm} András Zemes
		
    \item [Trieda:]	
    	\tabto{5cm} 4. IT
    	
	\item [Konzultant:]		 	  
		\tabto{5cm} Mgr. Peter Hudec
		
	\item [Školský rok:] 
		\tabto{5cm} 2018/2019
		
	\item [Odbor:]		  
		\tabto{5cm} Informačné a sieťové technológie
		
	\item [Názov témy:]			  
		\tabto{5cm} Jednozvodový elektrokardiograf 
		
	\item [Úloha:]				  
		\tabto{5cm} Zhotoviť prístroj, \emph{elektrokardiograf}, na snímanie 
		\tabto{5cm} a zachytenie elektrických potenciálov srdca.
	
	\item [Praktický charakter úlohy:]
		\tabto{5cm} Návrh plošného spoja, programovanie 
		\tabto{5cm} mikrokontroléra, vytvorenie grafickej aplikácie.
		
\end{description}
	
\vspace{20mm}
\hrulefill
\\\hspace*{0mm}\phantom{v.r.: }András Zemes, riešiteľ

\vspace{10mm}
\hrulefill
\\\hspace*{0mm}\phantom{v.r.: }Mgr. Peter Hudec, interný konzultant

\vspace{10mm}
\hrulefill
\\\hspace*{0mm}\phantom{v.r.: }Zástupkyňa riaditeľa školy

\vspace{10mm}
V Nových Zámkoch dňa \today

\newpage

\subsection*{Čiastkové úlohy}
\begin{enumerate}
	\item Naštudovanie a spracovanie potrebnej teórie
	\begin{itemize}
		\item Elektrický potenciál srdca
       	\item Prehľad prístrojov EKG
       	\item Signál a jeho spracovanie
	\end{itemize}
    \item Vytvorenie a konštrukcia prístroja na meranie EKG
    \item Vytvorenie elektronickej časti, práca s mikrokontrolérom
    \item Vytvorenie PC aplikácie na grafické zobrazenie spracovaných údajov
    \item Webové rozhranie k spracovaným dátam
\end{enumerate}

\newpage
\vspace*{\fill}
\section*{Čestné vyhlásenie}

Ja, dolupodpísaný András Zemes, študent 4. IT triedy Strednej priemyselnej školy S. A. Jedlika v Nových Zámkoch, týmto vyhlasujem, že som túto prácu   vyhotovil sám, s použitím uvedenej literatúry a podľa rád môjho konzultanta. 

\vspace{10mm}
\hrulefill
\\\hspace*{0mm}\phantom{v.r.: }András Zemes

\newpage
\vspace*{\fill}
\section*{Poďakovanie}
Touto cestou by som sa chcel poďakovať všetkým, ktorí mi akýmkoľvek spôsobom pomohli a povzbudzovali ma pri vypracovaní mojej komplexnej maturitnej práce. Predovšetkým však patrí moja vďaka konzultantovi, Mgr. Petrovi Hudecovi, za jeho všestrannú pomoc, za vedenie a cenné pripomienky pri záverečnom spracovaní práce.

\newpage
\tableofcontents

\newpage
\section{Úvod}
Ľudské telo je zázračný živý organizmus, ktorý sa správa podľa zákonitostí prírody a biológie. Dnes známa podoba Homo sapiens je výsledkom prirodzeného evolučného výberu, ktorým prechádza už milióny rokov. Vďaka nemu sú naše orgány vyspelé a odolné, dokonale slúžia prežitiu. Ich rola a presný spôsob fungovania však dlho zostávali záhadou pred lekármi a vedcami v minulosti. Revolučné objavy a výdobytky v medicíne viedli k podrobnému zmapovaniu \linebreak a poznaniu ľudského tela, i keď mnoho fenoménov je doposiaľ nevysvetlených. Choroby sa stali liečiteľnými a predĺžil sa predpokladaný vek dožitia.

V modernej dobe sa výrazne zmenil štýl, akým žijeme. Jeho dôsledky nesú naše telá, ktoré neboli stavané na rušný, uponáhľaný spôsob života a na zvládanie každodenného stresu. Nesprávna životospráva, zlé návyky a degradácia životného prostredia majú záporný vplyv na zdravotný stav obyvateľstva a prispievajú k šíreniu civilizačných ochorení. Jedná sa o degeneratívne ochorenia, ktoré patria medzi najpálčivejšie globálne zdravotnícke problémy. \linebreak V rebríčkoch najčastejších príčin smrti sa často vyskytujú na prvých priečkach.

Škodlivé zvyklosti ako nezdravé stravovanie, nedostatok fyzickej aktivity, fajčenie a nadmerná konzumácia alkoholu sa odzrkadľujú aj na srdci a obehovej sústave. Tieto a mnoho ďalších faktorov zvyšujú riziko srdcovocievnych ochorení vrátane kôrnatenia tepien, infarktu myokardu, vysokého krvného tlaku, atď. Ľudia si veľakrát ani neuvedomujú, že sa u nich vyvíja takáto choroba, alebo si to uvedomia neskoro.

Riešenie môže poskytnúť bioinformatika a biomedicínske inžinierstvo. Sú to vedné disciplíny, ktoré sa zaoberajú zhromažďovaním a vyhodnocovaním biologických dát a konštrukciou klinických zariadení. Vo veľkej miere uľahčujú prácu lekárom vo včasnom rozpoznaní a liečbe zdravotných problémov. Klasickým príkladom biomedicínskeho prístroja je elektrokardiograf, alebo skratkou EKG. 

V roku 1903 Willem Einthoven zostrojil prvý funkčný elektrokardiograf. Použil strunový galvanometer na snímanie elektrických potenciálov z končatín pacienta. Jeho vynález sa stal základným kameňom elektrokardiografie, za ktorý neskôr získal Nobelovu cenu za fyziológiu alebo medicínu. Vďaka Einthovenovi lekári dostali prvýkrát v histórii možnosť nahliadnuť hlboko do srdca a odhaliť skryté defekty prevodového systému.

Myslím si, že najväčším nepriateľom ľudstva je neinformovanosť. Pokiaľ chceme predísť chorobám a žiť zdravý, plnohodnotný život, musíme poznať možné následky rizikového správania. Zámerom tejto práce je šíriť povedomie o dôležitosti starostlivosti o zdravie a preukázať, že pozorovanie vnútrotelových javov nemusí byť tak zložité ako sa na prvý pohľad zdá. 



\newpage
\section{Ciele}
Hlavným cieľom práce je zhotoviť prístroj, elektrokardiograf, na snímanie a zachytenie elektrických potenciálov srdca. Zariadenie bude somatické ukazovatele získavať, zhromažďovať, filtrovať a poskytovať koncovému používateľovi v grafickej podobe na vyhodnotenie.

Kľúč úspešného zvládnutia projektu medzidisciplinárneho charakteru spočíva v dokonalej spolupráci jeho rôznych zložiek. Zámerom tejto práce je poukázať na to, že aj pomerne jednodu-chými súčiastkami, ktoré sa navzájom dopĺňajú a spolupracujú medzi sebou, sa dajú vyriešiť zložité problémy.

Výstupom práce bude prehľad relevantnej teórie spracovaný v dokumentácii, 
funkčné zaria-denie EKG s potrebným príslušenstvom, počítačový program a webová stránka, ktorá bude obsahovať dôležité informácie o projekte a ponúkať návštevníkom možnosť zdieľania nahratých záznamov EKG.

Medzi prioritami pri návrhu elektronických zložiek je na prvom mieste bezpečnosť, keďže sa pri prevádzke bude obvod bezprostredne a vodivo spájať s telom vyšetrovaného. Takýto kontakt je nevyhnutný kvôli potrebe snímania elektrických potenciálov.

Na základe výsledkov merania jednozvodovým EKG je možné určiť pulz, srdcový rytmus, ba aj odhaliť prítomnosť srdových arytmií a fibrilácie predsiení. Projekt môže taktiež slúžiť ako pomôcka pri výučbe elektroniky, programovania či informatiky. Znázorňuje fungovanie signálových filtrov, operačných zosilňovačov, mikrokontrolérov a grafických počítačových aplikácií.

Dôležité vlastnosti výslednej práce budú bezpečnosť, presnosť a viacúčelovosť. Prístroj by mal dokázať použiť každý bez zvláštneho vybavenia a bez podrobnej znalosti jeho fungovania. Riešenie má byť taktiež kompaktné, prenosné a miestom neviazané. 

\newpage
\section{Problematika a prehľad literatúry}
Kapitola uvádza základné poznatky a zaoberá sa teoretickými východiskami, ktoré s nastolenou problematikou súvisia. Princípy elektrokardiografie tvoria základ pre pochopenie činnosti zariadenia a sú jeho neoddeliteľnou súčasťou.


\subsection{Elektrofyziológia srdca}

Základom vnútorného fungovania srdca je jeho elektrická aktivita. Srdce je jedinečný orgán z pohľadu, že jeho elektrická činnosť nie je nervovo založená. Vykonávajú ju špecializované vodivé svalové bunky. Zväzky takýchto buniek určujú čerpaciu schopnosť srdca.

Svalové kontrakcie sú riadené elektrickými impulzmi, ktoré sa šíria po prevodovom systéme a pracovnom myokarde. Menia elektrické potenciály na rôznych bodoch pokožky približne \linebreak o tisícinu voltu (1 mV). Táto elektrická aktivita skrýva v sebe neuveriteľné množstvo informácií, prostredníctvom ktorých získavame náhľad do fungovania srdca.

Zdrojom týchto impulzov je sinoatriálny (SA) uzol, ktorý sa nachádza v stene hornej časti pravej predsiene. Udáva frekvenciu kontrakcií myokardu (srdcového svalstva), ktorá je nomi-nálne 70 tepov za minútu.

Signály sa ďalej šíria vodivými dráhami predsiení a stimulujú svalové kontrakcie. Pokračujú po srdcovej priehradke, septe, ktorá oddeľuje dve polovice srdca. Blízko bodu spojenia štyroch dutín srdca sa nachádza zhluk špeciálnych buniek - atrioventrikulárny (AV) uzol. Uzol AV postup vzruchov mierne spomaľuje a následne ich vysiela do Hisovho zväzku, ktorý sa delí na dve vetvy, tzv. Tawarove ramienka. Obidve vetvy vedú do siete Purkyňových vláken, ktoré aktivujú pracovný myokard.
\\
\\

\begin{figure}[!ht]
\begin{center}
\includegraphics[scale=0.7]{48_img1_b}
\caption{Prevodový systém srdca}
\end{center}
\end{figure}

\newpage
\subsection{Akčný potenciál}
Ako vyrába a prenáša srdcové tkanivo elektrické impulzy? Pre pochopenie tohto deja sa musíme preniesť až na úroveň atómov.

\emph{Atóm je neutrálny,} ak má rovnaký počet protónov (kladne nabitých častíc) a elektrónov (záporne nabitých častíc). 

\emph{Ióny vznikajú} z elektricky neutrálnych atómov pridaním resp. ubraním elektrónov. 

V pokoji sa srdcová bunka nachádza v \emph{polarizovanom stave}:
\begin{itemize}
	\item mimobunkový priestor je elektricky pozitívny pre vysokú koncentráciu kladných iónov sodíka a vápnika
	\item vnútrobunkový priestor je oproti vonkajšej strane elektricky negatívny
	\item rozdiel potenciálov medzi vnútrobunkovým a mimobunkovým priestorom je 90mV
\end{itemize}

Keď pokojový membránový potenciál dosiahne určitú prahovú hodnotu (cca. 15 mV), tzv. \emph{akčný potenciál}, tento pokojový stav sa náhle zmení. V membráne bunky sa otvoria prieduchy a kladne nabité ióny prúdia späť dovnútra bunky. Táto náhla strata polarizácie sa volá depolarizácia a vzniká pri nej elektrický prúd.

Po depolarizácii nastáva protikladný dej, repolarizácia, keď sa ióny znovu prečerpávajú von mimo membránu. Depolarizačná vlna vyvolávaná uzlom SA sa šíri po prevodovom systéme srdca a uvádza svaly do pohybu. Proces, ktorý začal pumpovaním iónov takto končí pumpovaním krvi. 

\begin{figure}[!ht]
\begin{center}
\includegraphics[scale=0.5]{action-potential-voltage}
\caption{Akčný potenciál v grafickom vyobrazení}
\end{center}
\end{figure}

%https://www.jfmed.uniba.sk/fileadmin/jlf/Pracoviska/ustav-patologickej-fyziologie/07Pregradualne_studium/01Vseobecne_lekarstvo/04Handouty_a_prednasky/01Handouty/01Elektrokardiografi1-jun10.pdf


\newpage 
\subsection{Prehľad prístrojov EKG}
\subsubsection{Druhy funkčných vyšetrení}
Elektrokardiografia je jedným zo základných lekárskych vyšetrení. Najčastejšie sa využíva \linebreak v núdzových situáciách pri podozrení na srdcový infarkt, na zistenie poruchy súvisiacej s kardiovaskulárnym systémom alebo ako preventívne vyšetrenie so zámerom odhaliť možný srdcový defekt.

Vyšetrenie EKG je neinvazívne a nevyžaduje žiadnu špeciálnu prípravu. Pri klasickom EKG sa elektródy pripevnia na hrudník, zápästia a členky pacienta. Elektrické signály zachytené \linebreak z povrchu tela, ktoré sú spravidla veľmi slabé, v rádoch milivoltov, prístroj zosilní a zaznamená. Následne ich lekár vyhodnotí.

Existuje niekoľko rôznych druhov vyšetrní:
\subsubsection*{Štandardné 12-zvodové EKG}
Je najčastejšie používaný zo všetkých typov EKG. Pozostáva zo 6 končatinových zvodov a 6 hrudných zvodov. Každý zvod je samostatne zapisovaný na priebežne sa posunujúci špeciálny záznamový papier, prípadne zobrazuje hodnoty na monitore.
\subsubsection*{Záťažové EKG (ergometria)}
Ukáže správanie srdca a obehového systému pri námahovej aktivite. Na simuláciu sa väčšinou používa stacionárny bicykel alebo bežecký pás. Monitoruje sa záznam EKG v súvislosti \linebreak s krvným tlakom. Na zázname sa pátra po zmenách, ktoré na EKG urobenom v pokoji nie sú viditeľné.
\subsubsection*{Dynamické EKG}
Umožňuje sledovať srdcovú činnosť pri bežných aktivitách počas 12-48 hodín. Zvýšením doby monitorovania sa zvyšuje pravdepodobnosť nálezu nepravidelností rytmu alebo námahových ischémií myokardu\footnote{Ischémia myokardu - nedostatočné prekrvenie srdcového svalu} v zázname.
\\
\\
EKG určuje základné fyziologické hodnoty ako sú frekvencia srdcovej činnosti, rytmus, elektrická os srdca, prevodové časy a morfológia segmentov EKG krivky. Na základe týchto para-metrov môže byť stanovená diagnóza a rozpoznaná porucha srdcového rytmu (arytmia), porucha prevodu elektrických vzruchov, ischemická choroba srdca a iné patologické zmeny v myokarde.


\newpage


\subsubsection{Časti klasického prístroja EKG}

\textbf{Tepelná tlačová hlava}
\\
Kreslí EKG krivku generovaním tepla.
\\
\\
\textbf{Termopapier}
\\
Prichádza do kontaktu s tlačovou hlavou. Na mieste dotyku sa farba papiera mení na čiernu, takto vzniká krivka EKG. Papier je tiež citlivý na tlak.
\\
\\
\textbf{Elektródy}
\\
Elektródy sú vyrobené z vodivého materiálu, ktorý dokáže zachytiť elektrické impulzy zo srdca. Signály odosielajú na spracovanie do meracieho prístroja cez pripojené káble.
\\
\\
\textbf{Zosilňovač}
\\
Zosilňovač je zariadenie, ktoré sa nachádza v elektrokardiografe a zvyšuje amplitúdu elektrického signálu. Signály prichádzajúce zo srdca sú relatívne slabé (0,0001V až 0,003V) a je potrebné ich zosilniť.
\\
\\
\textbf{Galvanometer}
\\
Premieňa prúd na mechanický pohyb.
\\
\\
\textbf{Sada EKG káblov}
\\
Slúžia na spojenie zvyčajne desiatich elektród\footnote{Kombináciou dvojíc elektród EKG prístroj na papier vygeneruje EKG zvody} s hlavnou jednotkou prístroja EKG. Takáto konfigurácia umožňuje monitorovať srdce z 12 ,,pohľadov". Vodiče medicínskych prístrojov bývajú spravidla chránené zvýšeným tienením pred vysokofrekvenčným rušením.

\subsubsection{Výdobytky modernej elektrokardiografie}
Moderné prístroje EKG disponujú zabudovanými mikroprocesormi, ktoré ich riadia a rozširujú ich diagnostické schopnosti. Vďaka sofistikovaným matematickým algoritmom a modelom sú schopné previesť zložitú analýzu signálu a automaticky ho vyhodnotiť. Sú kompaktné, prenosné a vhodné i na monitorovanie mimo zdravotníckeho zariadenia.

Prepojiteľnosť s počítačom je v dnešnej dobe takmer samozrejmosťou, niektoré dokonca komunikujú bezdrôtovo a aj na diaľku. Digitalizácia údajov môže byť výhodná napríklad z hľadiska archivácie alebo v prípade potreby zdieľať záznam so špecialistom.

Kardiologický monitor je častým rozšírením zariadenia EKG a umožňuje dlhodobo sledovať srdcovú aktivitu pacienta. Údaje zobrazuje v reálnom čase a ponúka náhľad kriviek ešte pred ich zápisom na papier.


\newpage

\subsection{Umiestnenie elektród}
Tri končatinové elektródy (pravá ruka, ľavá ruka, ľavá noha)  vytvárajú Einthovenov trojuholník. Vzniknú 3 \emph{bipolárne zvody} reprezentované stranami trojuholníka. Každý zvod pozostáva \linebreak z dvoch elektród, z jednej pozitívnej a z jednej negatívnej. Negatívna elektróda predstavuje depolarizovanú časť (-), pozitívna repolarizovanú časť (+) srdca. Pozitívny a negatívny pól spolu tvoria elektrický vektor, ktorý sa premieta na papier ako krivka EKG.

\begin{description}
	\item Zvod I: \tabto{1cm} $V_{I} = \phi_L - \phi_R$
	\item Zvod II: \tabto{1cm} $V_{II} = \phi_F - \phi_R$
	\item Zvod III:	\tabto{1cm} $V_{III} = \phi_F - \phi_L$
\end{description}
, kde: \\
\tabto{1cm} $V_{I}$ = napätie zvodu I\\
\tabto{1cm} $V_{II}$ = napätie zvodu II\\
\tabto{1cm} $V_{III}$ = napätie zvodu III\\
\tabto{1cm} $\phi_L$ = potenciál na ľavej ruke\\
\tabto{1cm} $\phi_R$ = potenciál na pravej ruke\\
\tabto{1cm} $\phi_F$ = potenciál na ľavej nohe\\
\\
Podľa Kirchhoffovho zákona platí, že veľkosť potenciálov (amplitúd na EKG zázname) vo zvode $V_{II}$ je sumou potenciálov v zvodoch $V_{I}$ a  $V_{III}$:
\tabto{1cm} $V_{I} + V_{III} = V_{II}$
, z čoho vyplýva, že iba dva z troch zvodov sú nezávislé.
\\
\\
Einthoven definoval rozdiely potenciálov medzi troma pármi horeuvedených bodov ako základné končatinové zvody v elektrokardiografii.


\begin{figure}[!ht]
\begin{center}
\includegraphics[scale=0.4]{ekg-zvody}
\caption{Einthovenove končatinové zvody a Einthovenov trojuholník}
\end{center}
\end{figure}
%http://www.bem.fi/book/15/15.htm


\subsection{Signál a jeho spracovanie}
Po úspešnom zmeraní a zosilnení signálu je potrebné vysporiadať sa s určitými prekážkami, nepresnosťami v snahe zachytiť srdcový rytmus. Signál je síce zosilnený, ale naďalej obsahuje mnoho nežiaducich elementov vplyvom rušivých faktorov z okolia. Výsledkom je skreslený biosignál, ktorý je v tejto fáze nepoužiteľný.

Nepresnosti v meraniach odborne nazývame \emph{artefakty}. Artefakty sa môžu prejavovať \linebreak v menšej či väčšej miere v závislosti od nedokonalostí v priebehu vedenia signálu z pacienta do aparatúry (prístroja). V elektrokardiografii rozoznávame tri základné druhy artefaktov:
\begin{itemize}
	\item sieťový brum
	\item kolísanie nulovej línie (drift)
	\item myopotenciály
\end{itemize}

V minimalizácii nežiaduceho šumu nám napomáha súbor špecializovaných hardvérových \linebreak i digitálnych filtrov.

\subsubsection{Sieťový brum}
Prvým krokom spracovania signálu je základná hardvérová filtrácia. Elektromagnetická interferencia (EMI) vzniká pôsobením  elektromagnetického poľa z elektrickej siete. Pri tomto jave dochádza k vzniku indukovaného napätia (Ui) a indukovaného prúdu vo vodiči. Šum opísaného druhu môžeme charakterizovať pri frekvencii 50 Hz sínusového rušenia. Na potlačenie sieťového brumu je účinná kombinácia hardvérového RC článku s digitálnym filtrom. 

\subsubsection{Potlačenie driftu}
Drift alebo kolísanie nulovej línie opisuje skupinu elektrochemických a mechanických javov. Príkladmi elektrochemických sú potenie pod elektródami, nedostatočné odmastenie pokožky, malé množstvo kontaktného gélu. Dýchanie (do 0,8 Hz) a pomalé pohyby klienta (do 2 Hz) sú mechanické javy. Na odstránenie nízkofrekvenčnej rušivej zložky použijeme hornopriepustný filter.

\subsubsection{Myopotenciály}
Ďalší rušivý faktor pri vyšetrení EKG predstavuje svalová aktivita, najmä pri záťažovom EKG. Svaly počas pohybu vytvárajú elektrické impulzy, ktoré sa potom prejavujú vo forme muskuloskeletálneho artefaktu\footnote{nepresnosť merania spôsobená pohybom}. Najväčším problémom v zdolaní účinku myopotenciálov je vzájomné prekrývanie frekvenčného pásma svalovej aktivity a užitočného pásma EKG. Na odstránenie tohto artefaktu nie je účinná pásmová priepusť. Vyžaduje sa pokročilejšie riešenie, napríklad pomocou adaptívnej filtrácie. 

\newpage
\subsubsection{Číslicové filtre}

\subsubsection*{Konečná impulzná odozva - FIR}
,,Filtre s konečnou impulznou charakteristikou (FIR – Finite Impulse Response) sú filtrami  matematicky  jednoduchšie  opísateľnými  ako  IIR  filtre,  pretože  sú  väčšinou realizované nerekur-zívne (nemajú spätnú väzbu). Pri nerekurzívnej realizácií má filter všetky póly v počiatku, čo mu zaručuje stabilitu. Medzi základné vlastnosti FIR filtrov je okrem stability aj možnosť získať lineárnu fázovú kmitočtovú charakteristiku v celom kmitočtovom rozsahu." \textsuperscript{[7]}
\begin{flushright}
(Tuka, 2015)
\end{flushright}

\begin{figure}[!ht]
\begin{center}
\includegraphics[scale=0.2]{fir-blokova-schema}
\caption{Bloková schéma FIR filtra}
\end{center}
\end{figure}

\subsubsection*{Nekonečná impulzná odozva - IIR}
,,Filtre s nekonečnou impulznou charakteristikou (IIR – Infinite Impulse Response) vznikli ako číslicové varianty analógových filtrov. Sú matematicky zložitejšie ako FIR filtre, pretože  sú  realizované rekurzívne  (vždy  obsahujú  spätnú  väzbu).  Oproti  FIR filtrom však nie je treba tak veľký počet oneskorovacích členov a preto je možné ho použiť aj na kratšie dáta (inicializácia nie je tak dlhá). IIR filtre majú všeobecne väčší počet zapojení ako FIR práve pre ich rekurzívnu časť." \textsuperscript{[7]}
\begin{flushright}
(Tuka, 2015)
\end{flushright}
\begin{figure}[!ht]
\begin{center}
\includegraphics[scale=0.2]{iir-blokova-schema}
\caption{Bloková schéma IIR filtra}
\end{center}
\end{figure}


\newpage
\section{Metodika práce}
\subsection{Návrh a konštrukcia hardvéru}
Proces návrhu hardvérových komponentov sa rozdelil do niekoľkých fáz kvôli \mbox{systematickosti}. Takýto spôsob práce umožnil priebežné testovanie a odhaľovanie možných chýb počas vývoja. Od začiatku až po finálny dizajn projekt prešiel tromi iteráciami. 

Na návrh elektroniky a dizajn plošných spojov sa používal grafický počítačový editor \linebreak Eagle (verzia 9.1.3).

\subsubsection{Zosilňovací obvod na vývojovej doske}
Prvý prototyp obvodu bol vyrobený podľa jednoduchej schémy. Pozostával iba z jedného operač-ného zosilňovača a zopár rezistorov. Z bezpečnostných dôvodov bola namiesto laboratórneho zdroja použitá 9V batéria na napájanie obvodu. 

Operačný zosilňovač LM741 slúži na zosilnenie nízkonapäťového vstupu z elektród prilože-ných na povrch tela. Je zapojený v diferenčnej konfigurácii, jeho invertujúci a neinvertujúci vstup predstavujú rozdielne napätia. Výstup je teda funkciou napäťovej diferencie medzi dvoma hrudnými elektródami. Faktor zisku je približne 50 podľa pomeru R5:R4. Z dôvodu, že zosilňo-vač funguje optimálne pre stredové hodnoty (medzi maximom a minimom), je nutné jeho vstupy dostať do použiteľného pásma. Na tento účel slúži napäťový delič R1-R2.

Analógový výstup bol pripojený do 3,5mm mikrofónového rozhrania zvukovej karty počítača. Signál prešiel základnou softvérovou filtráciou a bol graficky zobrazovaný pomocou počítačovej aplikácie.

Hlavným nedostatkom tohto návrhu bola všeobecná nespoľahlivosť a výskyt elektromagne-tickej interferencie a iných artefaktov v meraných hodnotách. Dôvodom bola mimoriadna náchylnosť kovových dráh dosky k rušeniu.


\begin{figure}[!ht]
\begin{center}
\includegraphics[scale=1]{schematic}
\caption{Schéma zosilňovacieho obvodu}
\end{center}
\end{figure}

\newpage
\subsubsection{Spájkovaný obvod}
Ďalším krokom bol návrh plošného spoja v Eagli podľa vyššej uvedenej schémy zosilňovacieho obvodu. Na základe počítačového návrhu bol zhotovený druhý prototyp na spájkovateľnej vývojovej doske, ktorý už riešil väčšinu nedostatkov prvotného prototypu. Elektronické súčiastky boli osadené a pospájané vodivými cestami z cínu. Doska mala celkovo 6 vývodov:
\begin{itemize}
	\item VCC (+5V) napájanie
	\item GND - zem
	\item OUT - analógový výstup
	\item Hrudný zvod č. 1
	\item Hrudný zvod č. 2
	\item Pravá noha
\end{itemize}
Signál bol meraný a spracúvaný za pomoci Arduina UNO. Výstup zo zosilňovača bol spojený \linebreak s analógovým pinom Arduina A0 (ADC). Vykresľovanie krivky EKG sa zrealizoval zabudova-ným nástrojom programu Arduino IDE, ktorý sa nazýva Serial Plotter.

Zobrazovaný signál bol do veľkej miery znečistený, preto bolo nutné použiť softvérový filter. Najjednoduchší filter vhodný na úlohu bol súčasťou oficiálnej knižnice Filters.h. \footnote{Dokumentácia knižnice je dostupná na webovom sídle [15] \textit{https://playground.arduino.cc/Code/Filters}}

Nasledovný kód je implementáciou dolnopriepustného filtra RC, ktorý slúži na vyhladenie signálu. Filter je nastavený na frekvenciu 50 Hz. Obsahuje deklaráciu premenných a inicializáciu triedy FilterOnePole.

\begin{lstlisting}
float filterFrequency = 5.0; 

FilterOnePole lowpassFilter(LOWPASS, filterFrequency);

while(true) {
  lowpassFilter.input(analogRead(INPUT_PIN));
}
\end{lstlisting}

Filter funguje na báze nekonečnej impulznej odozvy (IIR). Hlavnou výhodou je, že riešenie kladie veľmi nízke nároky na pamäť a výpočtovú kapacitu. Veľké pozitívum znamená taktiež jednoduchosť implementácie do projektu. Onedlho sa však prejavili problémy spôsobené vlastnosťami spomínaného filtra. Po úvodnom zachytení impulzov amplitúda postupne klesala, až kým nedosiahla nulu. Takéto správanie je nepriaznivým vedľajším účinkom IIR filtrov. 

\newpage
\subsubsection{Finálny obvod a rozširovacie moduly}
Výsledný produkt zahŕňa niekoľko dôležitých komponentov. Ústrednú časť štruktúry zariadenia tvorí osobitne navrhnutý a na mieru vyrobený plošný spoj. Táto doska priamo spája analógovú časť aplikácie s digitálnym signálovým procesorom. Ďalej je doplnená o dva rozširovacie \linebreak moduly a batérie.

\subsubsection*{Mikroprocesor}
Mikroprocesor je centrálnym prvkom zariadenia. Zodpovedá za digitalizáciu a spracúvanie príchodzieho signálu, riadenie a koordináciu jednotlivých elektrických článkov v rámci aplikácie a odovzdávanie informácií počítaču.

Pri výbere mikroprocesora sa prihliadalo na výkon, úspornosť, počet vývodov a na veľkosť dostupnej programovej pamäte. Nakoniec sa uprednostnil 8-bitový mikroradič Atmega328P z rodiny megaAVR od firmy Atmel. Vyhovel všetkým požiadavkám projektu, je vhodný na úlohu DSP a disponuje kvalitnou dokumentáciou i komunitnou podporou zo strany vývojárov. 
Podrobnosti o mikroprocesore sú opísané v tabuľke nižšie.\footnote{Informácie boli čerpané zo stránky výrobcu [11] \textit{https://www.microchip.com/wwwproducts/en/ATmega328p}}

\begin{table}[htb]
\begin{tabular}{ll}
\textbf{Parameter}              & \textbf{Hodnota}     \\
Typ programovej pamäte          & Flash                \\
Veľkosť programovej pamäte (KB) & 32                   \\
Max. rýchlosť CPU (MIPS)        & 20                   \\
Komunikačné periférie           & 1-UART, 2-SPI, 1-I2C \\
Tepelná tolerancia (C)          & -40 až 85            \\
Napájacie napätie (V)           & 1,8 až 5,5           \\
Počet vývodov                   & 28                  
\end{tabular}
\caption{Parametre mikroprocesora}
\end{table}

\subsubsection*{Napaľovanie zavádzača}
Zavádzač (angl. bootloader) mikroprocesora je prvý program, ktorý sa spustí pri každom štarte. Asistuje pri nahrávaní kódu do flash pamäte a čip sa stáva samoprogramovacím, čím zaniká potreba programátora.

Samostatný čip Atmega328P sa dodáva bez zavádzača, tým pádom sa priamo nedá programovať. Proces napaľovania bootloadera riadi medzičlánok zvaný in-system program (ISP). Na túto rolu vyhovuje aj doska Arduino UNO.

Prvým krokom bolo nahrať šablónu ArduinoISP na dosku. Potom sa prepojil mikročip \linebreak s Arduinom podľa špecifickej schémy zapojenia (viď. prílohu C). V nastaveniach sa zvolil typ dosky ,,Arduino Duemilanove alebo Nano" pre 16 MHz konfiguráciu. Za programátor sa \linebreak vybrala možnosť ,,Arduino as ISP".

Po dôkladnej kontrole všetkých nastavení sa spustilo nahrávanie zavádzača (Tools $\rightarrow$ Burn Bootloader).


\newpage
\subsubsection*{FT232RL programátor s Mini USB}
Adaptér USB–serial zabezpečuje programovanie čipov a komunikáciu s nimi. Komunikácia prebieha cez UART, tj. cez Rx a Tx piny. Úspešné nahratie programu má jednu podmienku: na čipe sa musí nachádzať zavádzač. Modul je kompatibilný s napätiami 3.3V a taktiež 5V, ladenie sa umožňuje pomocou prepínača. Zapojenie musí byť doplnené o kondenzátor (\SI{0,1}{\micro\F}) a o pull up rezistor na vývode DTR, aby resetovanie mohlo správne prebehnúť.

\begin{table}[htb]
\begin{tabular}{ll}
\textbf{FT232RL}  	& \textbf{Atmega328}   \\
DTR          		& RESET                \\
TXD			 		& RX                   \\
RXD					& TX				   \\
5V					& VCC				   \\
CTS (clear to send) & patrí do štandardu FT232, nie je nutné ho použiť \\
GND					& GND
\end{tabular}
\caption{Definície pinov}
\end{table}

\subsubsection*{Slot pre Micro SD kartu}
Slot umožňuje čítanie a zapisovanie údajov na Micro SD kartu. Prenos údajov sa uskutočňuje cez štandardné SPI rozhranie. Modul funguje s logickým napätím 3.3V, avšak vďaka zabudovanému regulátoru toleruje aj 5V. Tým pádom je plne kompatibilný so všetkými Arduino doskami a s príslušnou natívnou knižnicou \emph{SD.h} z ponuky Arduino IDE.

\subsubsection*{Výroba integrovaného plošného spoja}
Hotový digitálny návrh skompletizovaného plošného spoja bol odoslaný do výroby profesionál-nemu výrobcovi. K objednávke bol priložený súbor \textit{.brd} vyexportovaný z projektu Eagle (viď. prílohu A). Špecifikácie objednanej dosky: \\
\emph{Jednostranný plošný spoj so spájkovateľným ochranným lakom} \\
\textbf{Hrúbka dosky a medenej vrstvy:} 1.5 mm, \SI{35}{\micro\m} \\
\textbf{Rozmery PCB:} 65x50 mm \\
\textbf{Počet otvorov:} 87 ks \\

Po doručení objednanej dosky nasledovalo manuálne osadenie súčiastok. Ako prvé boli prispájkované pätice integrovaných obvodov: DIP28 pre mikroprocesor a DIP8 pre operačný zosilňovač. Použitie pätíc sa odporúča v prípade citlivých komponentov z dôvodu, aby sa predišlo ich možnému poškodeniu pri vysokých teplotách počas spájkovania.

Ďalej boli postupne osadené ostatné súčiastky ako kryštál 16 MHz, cievka, rezistory a kera-mické kondenzátory. Pri osadzovaní bipolárnych súčiastok (napr. svetelné diódy, elektrolytické kondenzátory) sa nesmie zabúdať na správne smerovanie kladných, resp. záporných pólov.

Naposledy boli prispájkované napájacie a vstupno-výstupné piny pre rozširovacie moduly \linebreak a elektródové vodiče.

\newpage
\subsubsection{Konštrukcia zariadenia}
Posledným krokom v práci s hradvérom bolo umiestniť všetky komponenty do škatuľky. Na tento účel padla voľba na univerzálny kryt HM-1553DGY\footnote{Technický výkres krytu so špecifikáciami a rozmermi sa nachádza na stránke distribútora [16] \\ https://www.tme.eu/sk/Document/bd6787db93f837d8972974413558d5cd/HM-1553DBK.pdf} od Hammmondu s rozmermi \linebreak 89x147x24 mm v tmavosivom prevedení. Výrobný materiál je plast ABS. Kryt sa skladá \linebreak z troch častí: z predného panelu a z vrchnej a spodnej časti, ktoré sa uzatvárajú štyrmi skrutkami. 

Napájacie puzdro na tri AA batérie bolo uložené pozdĺžne v zadnej časti krytu. K nemu bol pripevnený zvisle polohovaný USB—Serial modul. Na zadnej stene bol taktiež vytvorený otvor na rozhranie Micro USB.

Okraje hlavného plošného spoja boli upravené a tvarované tak, aby sa doska mohla prichytiť o jeden z podporných stĺpikov. Následne bola vyvŕtaná diera pre svetelnú indikačnú diódu \linebreak a vytvorený obdĺžnikový otvor pre vypínač. Na prednom paneli boli vyvŕtané diery s väčším priemerom na vsadenie troch laboratórnych konektorov.

Nakoniec boli všetky komponenty upevnené obojstrannou lepiacou páskou. Rozširovacie moduly, vypínač a konektory boli zapojené do riadiacej dosky či už priamo alebo prepojovacími káblami. Prepojovacie káble museli byť vyrobené na mieru kvôli požiadavkám na dĺžku a počet vodičov a tvar prípojky.


\begin{figure}[!ht]
\begin{center}
\includegraphics[scale=.3]{assembly-2}
\includegraphics[scale=.3]{assembly-1}
\caption{Nárys a pôdorys škatuľky s rozmermi}
\end{center}
\end{figure}

\newpage
\subsection{Tvorba softvéru}
Softvérová zložka tvorí druhú navýznamnejšiu časť tejto práce. Z hľadiska spracúvania, zobrazovania, vyhodnocovania a eventuálnej použiteľnosti signálu zohrávajú programové prostriedky dôležitú a nenahraditeľnú úlohu. Vývoj softvéru predstavoval komplexný proces, ktorý zahŕňal zužitkovanie možností rôznych programovacích jazykov a prostredí.

Každý stupeň práce so signálom má za úlohu riešiť samostatný problém a zároveň komunikovať s ostatnými jednotkami. Aplikačná vrstva projektu sa delí na tri hlavné časti:
\begin{enumerate}
	\item Hardvérové programovanie
	\item Grafické používateľské rozhranie (GUI)
	\item Webová aplikácia
\end{enumerate}

\subsubsection{Hardvérové programovanie}
Prvá fáza sa zaoberá primárnym filtrovaním na úrovni mikroprocesora. Pozostáva z potláčania sieťovového brumu, kolísania nulovej línie a iných artefaktov (viď. podkapitolu \textit{2.5 Signál \linebreak a jeho spracovanie}). Na tento účel boli použité zdrojové kódy z voľne dostupného internetového repozitára \footnote{Repozitár sa nachádza na adrese [14] https://github.com/tttapa/Filters}, keďže návrh filtrov nespadá do sféry pôsobnosti tejto odbornej práce. V kóde sú implementované knižnice filtrov FIR a IIR. Obsahuje taktiež matematicky odvodené koeficienty, ktoré určujú charakteristiky filtrov zvlášť prispôsobené elektrokardiografii. Softvér bol doplnený o algoritmus, ktorý popri filtrovaní súčasne riadi vysokorýchlostné zaznamenávanie údajov na pripojenú pamäťovú kartu MicroSD.

\subsubsection{Grafické používateľské rozhranie (GUI)}
Na zobrazenie zmeranej krivky EKG existujú tri možnosti. Jedna z nich je mechanický spôsob, čiže tepelná tlač na papier. Ďalšie sú integrovaná obrazovka alebo externé zariadenie s displejom. Najuniverzálnejšie a najekonomickejšie riešenie je signál spracovať na počítači. Pri voľbe vhodného prostredia na tvorbu softvéru sa prihliadalo i na multiplatformovosť.

Processing 3 je voľne dostupná grafická knižnica a vývojové prostredie (IDE), ktoré disponuje prostriedkami umožňujúcimi vykresľovanie grafov a priamu komunikáciou s mikrokontrolérom. Processing funguje na základe programovacieho jazyka Java a obsahuje ďalšie rozšírené technické a matematické nástroje. 

Sériová komunikácia prebieha pomerne vysokou prenosovou rýchlosťou 115200 bps pre dosíahnutie čo najväčšieho rozlíšenia. Prichádzajúce údaje odchytáva \textbf{sledovač udalostí (event listener)}, ktorý má za úlohu ich transformovať do zobraziteľného rozsahu.

O samotné vykresľovanie grafu sa stará natívna funkcia \textbf{draw()}, ktorá sa opakovane vykonáva počas behu programu. Definuje nové súradnice krivky grafu a spája ich s posledným známym bodom grafu.

Grafické užívateľské rozhranie zabezpečuje knižnica ControlP5 zvlášť určená pre vývojové prostredie Processing. Úpravou a rozšírením jej tried boli na mieru vytvorené ovládacie prvky.  

Funkcie programu zahŕňajú výber vstupného portu, ukladanie snímok grafu a exportovanie zaznamenaných hodnôt. Výber portu funguje pomocou rolovacieho menu (dropdown menu), ktoré obsahuje všetky dostupné rozhrania. Po zvolení portu, ak bol úspešne otvorený, sa začne vykresľovať graf na obrazovke. Naľavo od tejto lišty sa nachádza tlačidlo na zosnímanie obrazovky. Snímka sa následne uloží do predvoleného adresára. Tretia možnosť je nahrávanie meraných hodnôt prostredníctvom prepínača ,,REC". Po ukončení záznamu - opätovným klik-nutím na prepínač - sa vygeneruje súbor CSV.

\subsubsection{Webová stránka}
Webová stránka bola vytvorená pomocou frontendového nástroja Bootstrap. Je plne responzívna, prispôsobuje sa obrazovke na mobilných zariadeniach. Okrem poskytovania základných kontaktných údajov a stručného popisu projektu umožňuje aj vizualizáciu zaznamenaných dát EKG. Pracuje so súborovým formátom CSV, ktorý je totožný s typom exportovaným z počítačovej aplikácie a ukladaným na SD kartu z mikročipu. Po nahratí súboru sa dáta posielajú prostredníctvom JavaScript API internetovej službe na vyhotovenie grafu\footnote{Oficiálna stránka API je https://canvasjs.com/}, ktorý sa nakoniec zobrazí používateľovi. 


\newpage
\section{Výsledky a diskusia}
Komplexná odborná práca sa zaoberá zhotovením zariadenia na získavanie, zhromažďovanie a spracúvanie somatických ukazovateľov, presnejšie elektrických impulzov srdca. Prezentuje funkčné riešenie prejednávanej problematiky a vzťahujúcej sa teórie. Výstupy projektu sú výsled-kom niekoľkomesačného výskumu, dizajnu, prototypovania a testovania rôznych technických alternatív.

Výstupy práce sú nasledovné:
\paragraph{\textbf{Kvantitatívne}}
\begin{enumerate}
	\item Tvorba a návrh nezávislého, účelového hardvéru
	\item Grafická počítačová aplikácia
	\item Interaktívne webové rozhranie
	\item Podrobná dokumentácia problematiky a pracovného postupu
\end{enumerate}
\paragraph{\textbf{Kvalitatívne}}
\begin{enumerate}
	\item Dostupné riešenie komplexného problému
	\item Prospešné premostenie rôznych odborov: zdravotníctva, elektroniky a informatiky
	\item Aplikácia a presadenie teórie EKG do praxe
	\item Oboznámenie verejnosti s možnosťami EKG
\end{enumerate}


\paragraph{Úlohou zariadenia je sledovanie srdcového rytmu.} Kombináciou rôznych postupov a úprav vzniká žiadaný graf. Prvotne musí zosnímaný signál prejsť niekoľkonásobným zosilnením. Následne je prevedený do digitálnej podoby na ďalšie spracovanie. Prijíma ho signálový procesor, ktorý má na starosti odfiltrovanie rušivých činiteľov. Tento účinok sa dosahuje za pomoci číslicových filtrov a matematických modelov. Výsledok sa buď priamo zapisuje na dátový nosič, alebo sa prenáša do počítača. V prípade druhej alternatívy dáta v reálnom čase vykresľuje počítačová aplikácia. Taktiež rožširuje možnosti práce so signálom vďaka dostupným programovým prostriedkom. Užívatelia budú mať príležitosť dáta prehliadať a zdieľať cez webové rozhranie.

Vizuálna schéma popísaného postupu je uvedená nižšie v podobe blokového diagramu. \linebreak Vstupné údaje sú na ňom znázornené šípkou v ľavom hornom rohu. Šípka smeruje k prvému rámcu (bloku), ktorý zastupuje samostatný krok v reťazi úkonov. Tento sled úkonov zabezpečuje premenu údajov na informácie a vedie k výstupným bodom, akými sú napríklad pamäťová karta a grafické rozhranie počítačovej alebo webovej aplikácie.


\newpage
\begin{figure}[!ht]
\begin{center}
\includegraphics[scale=0.7]{flowchart}
\caption{Bloková schéma spracovania signálu}
\end{center}
\end{figure}

\bigskip

\textbf{Bezpečnosť}

Zariadenie nie je určené na masové použitie, preto jeho bezpečnosť nie je oficiálne zaručená. Pri návrhu boli zohľadnené možné riziká pri prevádzke zariadenia a implementovali sa primerané bezpečnostné opatrenia. 

Najdôležitejšou ochranou je vylúčenie napájania laboratórnym zdrojom. Pri nesprávnom nastavení alebo poruche zdroja by sa nebezpečný prúd mohol dostať do kontaktu s telom vyšetro-vaného. Aby sa podobnému scenáru zabránilo, obvod napájajú tri AA články (4,5V). 

Pri jednosmernom prúde má najväčší význam elektrolytický účinok, pretože väčšie elektrochemické zmeny dráždia pohybové nervy, čím spôsobujú kŕčovité stiahnutie svalov. Pri veľkých prúdoch sa zastavuje životná činnosť buniek.

Jednosmerné prúdy do 3 mA nevyvolávajú obyčajne žiadny pocit. Pri 5-10 mA sú sprievodné znaky svrbenie a pocit tepla. Ak prúd prechádza hrudníkom, pri prúdoch 80-100 mA je dýchanie takmer nemožné, komory srdca sa začnú kŕčovito sťahovať (fibrilácia komôr) a dochádza \linebreak k zástave krvného obehu.

Prípojné miesta tela na EKG:
\begin{itemize}
	\item Pravá noha - bod uzemnenia. Sám o sebe nepredstavuje nebezpečenstvo. Má rovnaký efekt ako napríklad použitie antistatického náramku.
	\item Hrudný zvod č. 1 - spojený s vysokoimpedančným vstupom zosilňovača a s batériou cez \SI{10}{\mega\ohm} rezistor (chráni pred nebezpečným prúdom).
	\item Hrudný zvod č. 2 - okrem vysokoimpedančného vstupu zosilňovača je tiež spojený \linebreak s výstupom cez \SI{100}{\kilo\ohm} rezistor. Treba si však uvedomiť, že tento prúd ani pri maximálnych hodnotách (4,5V) nepresahuje 0,045mA.
\end{itemize}

Bezpečnosť zariadenia by sa ďalej mohla vylepšiť oddelením optočlánkom. V tom prípade by sa dosiahla úplná separácia vstupných elektród od elektrickej časti.
\newpage

\textbf{Očistenie signálu od nežiaducich faktorov}

Vďaka kombinácii hardvérových a softvérových filtrov sa podarilo dosiahnuť vyhovujúci výsledok. Signál prechádza viacstupňovým filtrovaním. V rade filtrov  vystupuje Butterworth, lineárny filter a dve pásmové priepuste. 

Ďalšie spôsoby znižovania dopadu rušivých faktorov sú mechanického druhu. Pri meraní máme možnosť aplikovať vodivý gél, ktorý znižuje odpor medzi pokožkou a snímacími elek-tródami. Pokojová poloha v ľahu, dĺžka a tienenie káblov a zdroj napájania taktiež dokážu ovplyvniť konečný signál.

Spoločným efektom vymenovaných prevenčných prostriedkov je eliminácia artefaktov \linebreak rôzneho pôvodu. Na grafoch nižšie je zobrazená podoba signálu pred procesom filtrovania \linebreak a po ňom.

\begin{figure}[!ht]
\begin{center}
\includegraphics[scale=0.25]{ekg-emi}
\caption{Krivka EKG pred filtrovaním}
\includegraphics[scale=0.25]{ekg-chart}
\caption{Krivka EKG po viacstupňovom filtrovaní}
\includegraphics[scale=0.25]{ekg-artefact}
\caption{Artefakt na zázname EKG}
\end{center}
\end{figure}

\newpage

\textbf{Prínosy projektu a možnosti využitia}

Projekt svedčí o tom, že aj pomerne zložité oblasti, ako elektrokardiografia, môžu byť rele-vantné pre stredoškolskú úroveň. Jednou zo sfér, kde môže byť úžitočný je práve vzdelávanie. Práca sa zaoberá teóriou EKG a jej praktickým využitím. Demonštruje základné princípy fungovania dôležitého druhu zdravotníckeho príslušenstva.

Okrem toho umožňuje laikom sledovať pulz, odhaliť prítomnosť arytmií v pohodlí domáceho prostredia. Tým otvára nové možnosti na nelekárske použitie EKG. V prípade potreby sa výsledky dajú jednoducho zdieľať s odborníkom cez internet.
\\
\bigskip

\textbf{Obmedzenia riešenia}

Riešenie má samozrejme aj svoje nevýhody. Jedným z hlavných nevýhod je limitovanie počtu zvodov na jeden, čo výrazne obmedzuje množstvo informácií potrebných na stanovanie presnej diagnózy. Ďalej, keďže je hardvér minimalistický, rozlíšenie a kvalita merania nedosahuje úroveň komerčných lekárskych prístrojov.
\\
\bigskip

\textbf{Diskusia}

V akademickej sfére existuje nemálo študií a odborných prác, ktoré sa venujú práve téme elektrokardiografie. Toto medicínske odvetvie prešlo od jeho založenia pred vyše sto rokmi mnohými výraznými zmenami a vylepšeniami vďaka rozsiahlemu výskumu.

Tento projekt je praktického charakteru, nesnaží sa prekonať súčasné technológie a poznatky, ale prezentuje ich principiálnu podstatu dostupnými prostriedkami.  Zadané ciele boli splnené: (1) \textit{spracovanie vzťahujúcej sa teórie}, (2) \textit{návrh a konštrukcia prístroja EKG}, (3) \textit{programovanie mikrokontroléra}, (4) \textit{digitálne spracovanie údajov}.

Opísané riešenie merania EKG úvodným požiadavkam vyhovelo. Všetky nálezy sú zreprodukovateľné na základe dokumentácie pracovného postupu. Skutočnú výzvu týkajúcu sa danej problematiky však predstavuje navrhnutie prístroja spoľahlivého za každých okolností. Splnenie takéhoto nároku si vyžaduje robustnejšie medicínske riešenie a naďalej zostáva mimo dosahu používateľských aparátov - či už smart hodiniek alebo dosiek Arduino.

Funkcionalita EKG napriek tomu v poslednej dobe nadobúda popularitu a čoraz častejšie sa stáva súčasťou nositeľných zariadení, ktoré sledujú a vyhodnocujú každý náš krok. Stávame sa svedkami prekrývania záujmov technologického sektora a fitness priemyslu. Vďaka novým technologickým vymoženostiam sú nám informácie o zdraví a každodenných pohybových aktivitách dostupnejšie ako kedykoľvek predtým. Voľba, či sa vydáme smerom k zdravšej životospráve, je v našich rukách.



\newpage
\section{Závery práce}
Príležitosť na učenie sa ponúka každou ľudskou interakciou. Nové poznatky sa preverujú a ďalej utvrdzujú presadením do praxe. Je všeobecne známe, že sa človek najefektívnejšie učí skúšaním, z vlastných úspechov a chýb. Práve tieto vzostupy a pády sprevádzajúce naše pokusy nás večne ženú dopredu po ceste celoživotného vzdelávania.

Táto odborná práca možno nie je jediná svojho druhu, ale je výnimočná svojou mnoho-strannosťou. Zjednocuje diametrálne odlišné oblasti - zdravotníctvo a informatiku - a svedčí o prospechu ich prepojenia. Čerpá z vedomostných rezerv obidvoch odborov a opakovane poukazuje na množstvo križovatiek, ktoré ich spájajú. Práve tieto spoločné body ustanovujú predpoklady interdisciplinárneho dialógu.

Kľúčovou motiváciou k výberu témy bola i potreba upriamiť pozornosť na dôležitú rolu, ktorú hrá prevencia v zdraví jedinca. Napriek tomu, že mnohým chorobám sa predísť nedá, správna životospráva má ďalekosiahly dopad na imunitu a obranyschopnosť tela. Nepripúšťajme \linebreak si apatickosť, neoddávajme sa škodlivým návykom, veď zodpovednosť za životný štýl aký vedie-me nesieme sami.

Elektrokardiografia si v medicíne za posledných sto rokov jej existencie získala nespochybniteľný rešpekt. V zdravotníckych zariadeniach, ambulanciách, nemocniciach i v teréne zachra-ňuje životy každý deň. Dúfajme, že snahy o odkrytie niektorých nuáns za mechanizmom jej fungovania a o nález prieniku medzi odbornosťou a praktickosťou boli v konečnom dôsledku úspešné. 

Odkaz tejto práce dokonale vystihujú slová Reného Descarta: ,,Poznávanie pravdy je zdravie ľudského ducha.“

\subsection*{Resumé}
The objective of this project was to create a working single lead EKG. The work process was divided into several stages to provide clarity and a productive development framework. Firstly, the hardware had to be layed out, manufactured and constructed. Then followed the phase of software design, development and deployment on different application layers. The finished product encompasses both physical and intellectual components.

The device measures biopotentials using body-surface electrodes. After amplification the signal is processed and filtered by a programmable microchip. Next, the filtered data gets transferred into the computer, where a program plots out the resulting graph. The input stream can be saved for later use either as a visual snapshot or as a data file.

A lot can be learned from such a comprehensive, theoretically and practically challanging undertaking. Let the subject of this project be a testimony to the importance of interdisciplinary research and exchange of knowledge.





\newpage
\section{Zoznam použitej literatúry}
\begin{enumerate}[label={[\arabic*]}]
\item GEMINI, spol. s.r.o. 1991. Ľudské telo - Komplexný sprievodca po ľudskom tele a jeho funkciách. Bratislava. ISBN 80-85265-12-5.
\item IAIZZO, Paul A. 2005. Handbook of Cardiac Anatomy, Physiology, and Devices. New Jersey. Human Press, Inc. ISBN 1-59259-835-8.
\item HANÁČEK, Ján, PLEVKOVÁ, Jana. 2009. Elektrokardiografia - Základné mechanizmy porúch elektrickej funkcie srdca a ich manifestácia na Ekg krivke. Martin. Ústav patologickej fyziológie JLF UK.
\item MALMIVUO, Jaakko, PLONSEY, Robert. 1995. Bioelectromagnetism - Principles and Applications of Bioelectric and Biomagnetic Fields. Oxford. Oxford University Press.
\item BÓRIKOVÁ, Ivana. 2016. Funkčné vyšetrenie respiračného, kardiovaskulárneho a močového systému. Portál Jesseniovej lekárskej fakulty Univerzity Komenského. ISSN 1337-7396.
\item MIŠČÍK, Peter. 2011. Zpracování elektrokardiogramu. Vysoké učení technické v Brně. 
\item TUKA, Marián. 2015. Potlačení aditivního šumu. Vysoké učení technické v Brně. 
\item HARDEN, Scott. DIY ECG with 1 op-amp. SWHarden, [cit. 10. nov. 2018]. \\ URL: www.swharden.com/wp/2016-08-08-diy-ecg-with-1-op-amp/
\item BLAHÚT, Peter. Akčný potenciál. TECHmED, [cit. 19. dec. 2018]. \\ 
URL: https://www.techmed.sk/akcny-potencial/
\item HERMAN, Petř. Rušení biosignálů a artefakty. Wikisktripta,  [cit. 19. dec. 2018]. \\ URL: www.wikiskripta.eu/w/Rušení\_biosignálů\_a\_artefakty
\item Microchip Technology Inc. ATmega328P Device Overview. [cit. 14. jan. 2019]. \\ URL: https://www.microchip.com/wwwproducts/en/ATmega328p
\item KURNOTA, Marek. FT232RL programátor. Techfun, [cit. 21. jan. 2019]. \\ URL: https://techfun.sk/produkt/ft232rl-programator-s-mini-usb-5v-3-3v/
\item KURNOTA, Marek. Slot pre Micro SD kartu. Techfun, [cit. 21. jan. 2019]. \\ URL: https://techfun.sk/produkt/slot-pre-micro-sd-kartu/
\item PAS, Pieter. Filters. Github, [cit. 31. jan. 2019]. \\ URL: https://github.com/tttapa/Filters
\item A realtime digital signal processing (DSP) library. Arduino, [cit. 4. feb. 2019]. \\ URL: https://playground.arduino.cc/Code/Filters
\item Hammond Manufacturing. HM-1553DBK. TME, [cit. 14. feb. 2019] URL: \\https://www.tme.eu/sk/Document/bd6787db93f837d8972974413558d5cd/HM-1553DBK.pdf
\end{enumerate}

\newpage
\section{Prílohy}
\subsection*{Príloha A (obsah CD)}
\begin{itemize}
	\item Schéma zapojenia a návrh plošného spoja vyexportované z Eaglu
	\item Program nahratý na mikroprocesor a použité knižnice
	\item Zdrojový kód počítačovej aplikácie a spustiteľný súbor
	\item Zdrojový kód webovej stránky
\end{itemize}
\subsection*{Príloha B}
\begin{itemize}
	\item Fotodokumentácia projektu
\end{itemize}
\subsection*{Príloha C}
\begin{itemize}
	\item Schéma zapojenia Arduina pri nahrávaní zavádzača
\end{itemize}


\newpage 
\subsection*{Príloha B}
Fotodokumentácia projektu
\begin{figure}[!ht]
\begin{center}
\includegraphics[scale=0.3]{vyvojova-doska}
\\1. prototyp: Zosilňovací obvod na vývojovej doske
\end{center}
\end{figure}

\begin{figure}[!ht]
\begin{center}
\includegraphics[scale=0.08]{spajkovatelna-doska-1}
\includegraphics[scale=0.08]{spajkovatelna-doska-2}
\\2. prototyp: Spájkovaný obvod
\end{center}
\end{figure}

\begin{figure}[!ht]
\begin{center}
\includegraphics[scale=0.07]{skatulka-otvorena}
\includegraphics[scale=0.072]{skatulka-zatvorena}
\\3. prototyp: Finálny obvod
\end{center}
\end{figure}

\newpage
\subsection*{Príloha C}
Schéma zapojenia Arduina pri nahrávaní zavádzača
\begin{figure}[!ht]
\includegraphics[scale=70]{BreadboardAVR}
\end{figure}


\end{document}